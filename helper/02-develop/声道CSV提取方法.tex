\documentclass[12pt,a4paper]{article}
\usepackage{ctex}
\usepackage{amsmath,amssymb,amsfonts}
\usepackage{graphicx}
\usepackage{hyperref}
\usepackage{algorithm}
\usepackage{algorithmic}
\usepackage{listings}
\usepackage{xcolor}
\usepackage{geometry}

\geometry{a4paper,left=2.5cm,right=2.5cm,top=2.5cm,bottom=2.5cm}

\title{声道轮廓 CSV 提取方法}
\author{基于 3D Slicer 的实现}
\date{\today}

\begin{document}

\maketitle

\begin{abstract}
本文档总结了从 3D Slicer 中提取声道轮廓并转换为 VTL3D/Acoustic3dSimulation 所需 CSV 格式的方法。主要包括 n 个截面(2 个端点和 n-2 个中心线点)的切割、局部坐标系建立、轮廓点处理及规范化 CSV 输出等关键步骤。特别关注了轮廓方向、法线计算和坐标转换中的技术细节,以确保数据连贯性与正确性。
\end{abstract}

\section{背景}

声学研究中,声道三维模型的二维截面数据对声学仿真具有重要意义。本文介绍的方法从 3D Slicer 环境中导出的封闭腔体模型出发,按照特定格式提取截面轮廓信息,以便于后续声学模拟软件使用。

\subsection{CSV 文件规范}

根据 Acoustic3dSimulation::extractContoursFromCsvFile 函数的需求,CSV 文件格式定义如下:

\begin{itemize}
  \item 文件由成对的行组成,每对行代表一个声道横截面。
  \item 奇数行:包含全局 X 坐标和局部 Y 坐标相关数据:
    \begin{enumerate}
      \item 中心点全局 X 坐标 (double)
      \item 法线全局 X 分量 (double)
      \item 起始端缩放因子 (double)
      \item 轮廓点局部 Y 坐标序列 (double...)
    \end{enumerate}
  \item 偶数行:包含全局 Y 坐标和局部 Z 坐标相关数据:
    \begin{enumerate}
      \item 中心点全局 Y 坐标 (double)
      \item 法线全局 Y 分量 (double)
      \item 末端缩放因子 (double)
      \item 轮廓点局部 Z 坐标序列 (double...)
    \end{enumerate}
  \item 字段分隔符为分号 (';')
\end{itemize}

\section{方法概述}

整个提取过程可分为以下几个主要步骤:

\begin{enumerate}
  \item 组合端点与中心线,得到 n 个采样点
  \item 计算每个采样点的切向量(法线方向)
  \item 设置 vtkCutter 对每个点进行切片
  \item 建立局部坐标系并计算轮廓点局部坐标
  \item 按极角排序轮廓点
  \item 按规定格式写入 CSV
\end{enumerate}

\section{实现细节}

\subsection{中心线采样与切向量}

首先从两个端点(Endpoints)和中心线曲线(Centerline curve)读取控制点,组合为完整的采样点序列:

\begin{equation}
\text{pts} = [P_{\text{start}}, \text{curve\_pts}, P_{\text{end}}]
\end{equation}

对每个采样点,使用三点滑动平均计算切向量 $\vec{t}$:

\begin{equation}
\vec{v}_i = 
\begin{cases}
\vec{p}_{i+1} - \vec{p}_i, & i = 0 \\
\frac{\vec{p}_{i+1} - \vec{p}_{i-1}}{2}, & 0 < i < n-1 \\
\vec{p}_i - \vec{p}_{i-1}, & i = n-1
\end{cases}
\end{equation}

\begin{equation}
\vec{t}_i = 
\begin{cases}
\vec{t}_1, & i = 0 \\
\frac{\vec{v}_{i-1} + \vec{v}_i + \vec{v}_{i+1}}{3}, & 0 < i < n-1 \\
\vec{t}_{n-2}, & i = n-1
\end{cases}
\end{equation}

然后对每个 $\vec{t}_i$ 进行归一化,得到单位切向量,用作截面法线。

\subsection{局部坐标系建立}

对每个采样点构建局部坐标系 $(\vec{n}, \vec{t}, \vec{b})$,其中:

\begin{itemize}
  \item $\vec{n}$ 为截面法线(即切向量)
  \item $\vec{t}$ 为截面内的"局部 Y 轴"
  \item $\vec{b}$ 为截面内的"局部 Z 轴"
\end{itemize}

计算方法如下:

\begin{align}
\vec{n} &= \text{normalize}(\vec{t}_i) \\
\vec{r} &= 
\begin{cases}
(0,0,1), & \text{if } |n_z| < 0.9 \\
(0,1,0), & \text{otherwise}
\end{cases} \\
\vec{t}_{\text{tmp}} &= \vec{n} \times \vec{r} \\
\vec{t} &= \text{normalize}(\vec{t}_{\text{tmp}} \cdot [1, -1, 1]) \\
\vec{b} &= \vec{n} \times \vec{t}
\end{align}

其中,$\vec{t}$ 计算中的乘法 $[1, -1, 1]$ 用于翻转 Y 分量,确保法线在 CSV 中的正确表示。

\subsection{轮廓点提取与排序}

使用 vtkCutter 和平面方程提取每个采样点的轮廓点:

\begin{equation}
\text{plane}(\vec{x}) = \vec{n} \cdot (\vec{x} - \vec{p}) = 0
\end{equation}

对每个轮廓点计算局部坐标:

\begin{align}
\text{local\_y} &= (\vec{pts} - \vec{p}) \cdot \vec{t} \\
\text{local\_z} &= (\vec{pts} - \vec{p}) \cdot \vec{b}
\end{align}

为保证轮廓点顺序一致,根据极角排序:

\begin{align}
\theta_i &= \arctan2(\text{local\_z}_i, \text{local\_y}_i) \\
\text{sort\_idx} &= \text{argsort}(\theta_i)
\end{align}

可通过 clockwiseContour 参数选择顺时针或逆时针排序。

\subsection{CSV 格式输出}

将处理后的数据按规定格式写入 CSV:

\begin{itemize}
  \item 奇数行:$[P_x, t_x, \text{scale}, \text{local\_y}_0, \text{local\_y}_1, ...]$
  \item 偶数行:$[P_y, t_y, \text{scale}, \text{local\_z}_0, \text{local\_z}_1, ...]$
\end{itemize}

其中,$t_x$ 和 $t_y$ 分别是局部 Y 轴 $\vec{t}$ 的 X 和 Y 分量,表示截面内的参考方向。

\section{关键技术点}

\subsection{法线方向修正}

最初遇到的主要问题是截面的"法线"方向在 CSV 中的表示:
\begin{itemize}
  \item CSV 中的"法线"实际指的是截面平面内的一个基向量(而非法向量)
  \item 需要确保这个基向量的 Y 分量符号一致
\end{itemize}

最终通过在 compute\_reference\_axes 函数中对叉积结果的 Y 分量取反解决:
\begin{equation}
\vec{t} = \text{normalize}(\vec{n} \times \vec{r} \cdot [1, -1, 1])
\end{equation}

\subsection{轮廓点排序}

轮廓点必须按顺时针或逆时针排序。通过计算轮廓点在局部坐标系中的极角,再用 argsort 处理:
\begin{equation}
\theta_i = \arctan2(\text{local\_z}_i, \text{local\_y}_i)
\end{equation}

这确保了轮廓点在每个截面内的一致性,便于后续渲染和分析。

\subsection{缩放处理}

脚本支持多种缩放策略:
\begin{itemize}
  \item 可选将坐标单位从 mm 转换为 cm
  \item 可选使用等效半径归一化轮廓点坐标
  \item 控制是否基于截面面积计算缩放因子
\end{itemize}

\section{使用说明}

在 3D Slicer Python Console 中执行:
\begin{verbatim}
>>> exec(open('/path/to/slicer_extract_contour_csv.py').read())
\end{verbatim}

关键可调参数包括:
\begin{itemize}
  \item fiducialName, curveName, modelName:数据节点名称
  \item outputDir, csvName:输出位置
  \item clockwiseContour:轮廓点排序方向
  \item use\_cm\_unit:单位转换
  \item scale\_by\_radius:缩放策略
\end{itemize}

\section{总结}

本文档描述的方法实现了从 3D Slicer 模型到 VTL3D/Acoustic3dSimulation 所需 CSV 格式的准确转换。通过精心设计的局部坐标系建立、轮廓点排序和法线方向处理,解决了坐标转换和数据连贯性的难点。所得 CSV 文件可被下游声学仿真工具直接使用,为声道声学特性研究提供了基础数据。

\end{document} 