% !TeX program = xelatex
\documentclass[12pt]{article}
\usepackage[margin=2.5cm]{geometry}
\usepackage{xeCJK}
\usepackage{hyperref}
\usepackage{longtable}
\usepackage{array}
\usepackage{listings}
\lstset{basicstyle=\ttfamily,breaklines=true}

\title{miniSlicer 本地化流水线\\(用户指南 + 白皮书)}
\author{miniSlicer i18n 团队}
\date{\today}

\begin{document}
\maketitle

\begin{abstract}
本文旨在给\textbf{最终使用者}(而非脚本开发者)一个\emph{一站式说明}:\emph{为什么}需要本地化流水线、\emph{能解决什么痛点}、\emph{怎样一步步使用},以及\emph{如何按需扩展到新的替换场景}。同 时也为二次开发者附上架构细节与测试准则。
\end{abstract}

%────────────────────────────
\section{为什么需要这条流水线?}
\subsection{背景场景}
\begin{itemize}
  \item miniSlicer 及其扩展模块以英文界面为主,阻碍中文科研/教学场景推广。
  \item Qt Linguist 对大型脚本模块支持有限,且增量维护困难。
  \item LLM 出现后,批量高质量翻译成为可能,但缺少\textbf{安全、可回滚、可配置}的落地方案。
\end{itemize}

\subsection{核心痛点}
\begin{enumerate}
  \item \textbf{文件量大}:数千个 \texttt{.ui} + 数万行源码,手工不可行。
  \item \textbf{格式脆弱}:XML / C++ 任何格式错误都导致编译或运行崩溃。
  \item \textbf{需求多变}:今天翻译 \lstinline|<string>|,明天改模块 title,后天又要处理 \lstinline|tr("...")|。
  \item \textbf{缺乏增量}:UI 每次改动都需重跑,耗时且浪费 token。
\end{enumerate}

%────────────────────────────
\section{方案概览}
\begin{enumerate}
  \item \textbf{可逆占位符}:以少见字符"¥"+10 位数字包夹,保证翻译文本定位唯一可还原。
  \item \textbf{配置驱动(rules.yaml)}:把"\emph{要替换哪些字符串}"写进 YAML;脚本通用,不再硬编码正则。
  \item \textbf{四步流水线}:\textbf{标记}→\textbf{翻译}→\textbf{写回}→\textbf{清理与验证}。
  \item \textbf{增量与测试}:基于 git diff 仅处理新增英文;\lstinline|pytest| 单元测试验证规则。
\end{enumerate}

图~\ref{fig:pipeline} 给出整体数据流。

% TODO: 可加入 \\includegraphics[width=0.9\textwidth]{pipeline.pdf}
\begin{center}
\fbox{\parbox{0.9\textwidth}{\centering 占位符→LLM→sed→安全写回 的流水线示意图}}
\end{center}
\label{fig:pipeline}

%────────────────────────────
\section{一分钟上手(Quick Start)}
\begin{enumerate}
  \item \textbf{准备环境}
    \begin{enumerate}
      \item Python 3.9+ 或 Nix;
      \item 设置 \lstinline|OPENAI_API_KEY|;
      \item 可选:安装 \lstinline|xmllint|\ 进行 XML 校验。
    \end{enumerate}
  \item \textbf{克隆仓库 + 进入脚本目录}
    \lstinline|git clone ... && cd script|
  \item \textbf{运行命令}
  \begin{lstlisting}
make translate         # 一键翻译 UI + 源码(根据 rules.yaml)
make fix              # 兜底清除 ¥XXXXXXXXXX¥ 占位符
  \end{lstlisting}
  \item \textbf{查看效果}:启动 miniSlicer;若仍有英文,可查看 \texttt{tmp\_*/report.log} 确认规则是否遗漏。
\end{enumerate}

%────────────────────────────
\section{可配置规则(rules.yaml)详解}
\subsection{YAML 结构}
\begin{lstlisting}
- name: qt_ui_string          # 规则名称
  ext: [ui]                  # 文件后缀
  mode: xml_xpath            # 抽取方式
  xpath: .//string           # XPath 表达式

- name: python_title
  ext: [py]
  mode: regex
  pattern: "(?P<prefix>\\.parent\\.title\\s*=\\s*(?:_?\\(\\s*)?[\'\"])(?P<text>[^\'\"]+)(?P<suffix>[\'\"])"

- name: cxx_tr
  ext: [cpp,cxx,h,hpp]
  mode: regex
  pattern: "(?P<prefix>(?:::\\w+)?tr\\(\\s*[\'\"])(?P<text>[^\'\"]+)(?P<suffix>[\'\"])"
\end{lstlisting}

\subsection{自定义规则流程}
\begin{enumerate}
  \item 复制 \texttt{rules.yaml},修改或新增条目;
  \item 可运行 \lstinline|make dry_run RULES=my.yaml| 查看命中统计;
  \item 调整无误后再执行 \lstinline|make translate RULES=my.yaml|。
\end{enumerate}

可在\texttt{rules.yaml} 内同时声明 UI XPath 与源码正则,流水线一次性完成两类翻译,无需区分"UI/源代码"子命令。

%────────────────────────────
\section{流水线分步解析}
\subsection{Step 1 生成占位符}
脚本 \texttt{generate\_placeholders.py} 读取 \texttt{rules.yaml},对每条规则:
\begin{itemize}
  \item 按 \textbf{ext} 过滤文件;
  \item 根据 \textbf{mode} 调用 XML/XPath 或正则;
  \item 对需要翻译的文本加前缀 \verb|¥0000001234¥ | 并复制到临时目录;
  \item 统计命中数写入 report,方便快速review。
\end{itemize}

\subsection{Step 2 调用 LLM 翻译}
\begin{itemize}
  \item 将所有带占位符行拼成 \texttt{all\_strings.txt},按 \texttt{CHUNK\_SIZE} 分块;
  \item 使用统一 Prompt(可在 \texttt{translate\_prompt.txt} 自定义风格);
  \item 输出 sed 单行命令列表。
\end{itemize}

\subsection{Step 3 安全写回}
\begin{itemize}
  \item 解析 sed → {id: 中文} 映射;
  \item 对 XML 用 DOM,源码用正则,回写中文并保持格式;
  \item 使用 \texttt{xmllint / black / clang-format}(若存在)二次校验。
\end{itemize}

\subsection{Step 4 清理与验证}
运行 \texttt{fix\_placeholders.py} 根据规则定义的清理策略,确保仓库无遗留 \verb|¥数字¥|。若仍检测到→脚本报错退出 CI。

%────────────────────────────
\section{常见问题 \& 解决方案}
\begin{longtable}{p{5cm} p{9cm}}
\hline
症状 / 日志片段 & 可能原因与处理 \\ \hline
\texttt{Invalid API key} & 环境变量 \texttt{OPENAI\_API\_KEY} 未设置或过期。\\
XML ParseError & 翻译文本含未转义字符 \verb|&|;执行 \lstinline|make fix| 重新清理并验证。\\
regex 未命中 & 在 rules.yaml 新增/修正 \texttt{pattern}。\\
CI 报 "impure-derivations" & 记得在 Garnix/Nix 加上 \texttt{--extra-experimental-features impure-derivations}。\\
\hline
\end{longtable}

%────────────────────────────
\section{进阶扩展}
\begin{enumerate}
  \item \textbf{多语言输出}:规则可加 \texttt{lang: ja},脚本多次调用 LLM 生成多语目录。
  \item \textbf{术语表驱动}:Prompt 中注入术语表,保证专业词一致。
  \item \textbf{CI Bot}:失败自动在 PR 留评论列出未翻译行,方便 reviewers。
  \item \textbf{可视化报告}:生成 \texttt{html} 展示规则命中率、示例 diff。
\end{enumerate}

%────────────────────────────
\section{内部实现(给开发者)}
\begin{itemize}
  \item 脚本目录结构与调用关系图;
  \item 依赖列表(仅标准库 + 可选 \texttt{requests, tqdm} 等);
  \item 单元测试:\texttt{tests/test\_rules.py} 用示例文件断言标记/清理对等;
  \item 性能:对 10k 文件标记 <30s,写回 O(文件数)。
\end{itemize}

%────────────────────────────
\section{贡献指南}
\begin{enumerate}
  \item 提交新规则:需附 \texttt{examples/} 用例 + 单测;
  \item 代码遵循 \texttt{black 23.7};
  \item PR 模板见 \texttt{.github/};
\end{enumerate}

%────────────────────────────
\section{许可证与免责声明}
\begin{itemize}
  \item miniSlicer 基于 APL 2.0;脚本遵循 MIT。\newline
  \item 翻译内容由 OpenAI 生成,作者不保证绝对准确性,使用前请专业复核。
\end{itemize}

\end{document} 