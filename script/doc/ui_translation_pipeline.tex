% !TeX program = xelatex
\documentclass[12pt]{article}
\usepackage[margin=2.5cm]{geometry}
\usepackage{xeCJK}
\usepackage{hyperref}
\usepackage{longtable}
\usepackage{array}
\usepackage{listings}
\lstset{basicstyle=\ttfamily,breaklines=true}

\title{miniSlicer UI 翻译流水线技术报告}
\author{自动生成}
\date{\today}

\begin{document}
\maketitle

\section{场景与目标}

\begin{itemize}
  \item \textbf{场景}:miniSlicer 项目包含数千个 Qt Designer 生成的 \lstinline|.ui| 文件,需要将界面文字由英文批量翻译为中文。
  \item \textbf{目标}:
    \begin{enumerate}
      \item 在\textbf{全离线 / 本地仓库}条件下完成文本抽取、调用 LLM 翻译、写回文件,并保证 XML 格式合法。
      \item \textbf{最小侵入}:原始文件仅在最后一步被覆盖;中间过程使用临时目录,方便回滚。
      \item \textbf{可持续迭代}:脚本可重复执行,支持后续 UI 变动的增量翻译;流水线易复用到其他格式或项目。
    \end{enumerate}
\end{itemize}

\section{总体流程}

流水线由五个核心脚本组成,见表~\ref{tab:pipeline},执行顺序如同一条数据流:

\begin{longtable}{>{\raggedright\arraybackslash}p{1cm} >{\raggedright\arraybackslash}p{7cm} >{\raggedright\arraybackslash}p{8cm}}
\caption{UI 翻译流水线各步骤}\label{tab:pipeline}\\
\hline
\textbf{序} & \textbf{入口脚本} & \textbf{核心职责} \\
\hline
1 & \texttt{generate\_ui\_placeholders.py} & 递归扫描 \texttt{.ui} 文件,对需要翻译的 \texttt{<string>} 节点加上占位符 \verb|¥0000000123¥ | 并复制到 \texttt{tmp\_ui/};同时拼接大文件 \texttt{all\_strings\_tagged.txt} 供 LLM 输入。 \\
2 & \texttt{translate\_ui\_with\_openai.py} & 以 System Prompt + 大文件分块调用 OpenAI(可指定 \texttt{--base\_url} 代理),生成 sed 风格的替换脚本 \texttt{replace.sh}。 \\
3 & \texttt{apply\_xml\_translations.py} & 解析 \texttt{replace.sh},将中文文本填充回 \texttt{tmp\_ui/} 下对应节点,并按原相对路径写到源码目录。 \\
4 & \texttt{remove\_ui\_placeholders.py} & 在 \texttt{tmp\_ui/} 阶段去掉占位符后缀,保证写回的 XML 已无标识符。 \\
5 & \texttt{fix\_ui\_placeholders.py} & 兜底:在真正源码树再扫一次,防止极端情况残留占位符。 \\
\hline
\end{longtable}

\subsection*{Makefile 目标}
\begin{lstlisting}
$ make -C script all      # = translate
$ make -C script clean    # 删除 tmp_ui 等临时产物
$ make -C script monitor  # 查看 OpenAI token 等
$ make -C script fix      # 单独运行最后修复步骤
$ make -C script code_translate  # 源码 title/tr 等翻译
$ make -C script code_fix        # 清理 UI+源码占位符
\end{lstlisting}

\section{关键技术拆解}

\subsection{占位符标记策略}
\begin{itemize}
  \item 使用稀有字符 "¥" 包围,固定宽度 \verb|ID_WIDTH=10|,形如 \verb|¥0000000123¥ |。
  \item 正则匹配简单:\verb|^¥\d{10}¥\s?| 能快速定位。
  \item 与 UTF-8 中文不冲突;\texttt{id\_map.jsonl} 记录 \textit{ID--路径} 映射,方便调试。
\end{itemize}

\subsection{LLM 翻译交互}
\begin{itemize}
  \item \textbf{Prompt Engineering}:系统提示固定,强调医学影像软件本地化风格。
  \item \textbf{块切分}:环境变量 \texttt{CHUNK\_SIZE} 控制单次输入长度,脚本自动拆分与合并。
  \item \textbf{安全写回}:不直接 shell sed,而是解析为字典后用 XML DOM 写入,避免格式损坏。
\end{itemize}

\subsection{结构化写回}
\begin{itemize}
  \item 统一使用 \texttt{xml.etree.ElementTree} 读写,保持属性顺序。
  \item 仅对变更节点写文件,减少无意义 diff。
\end{itemize}

\subsection{校验与回滚}
\begin{itemize}
  \item 若本地安装 \texttt{xmllint},自动校验所有生成的 \texttt{.ui} 文件。
  \item \texttt{tmp\_ui/} 保留带占位符的中间文件,可回滚。
  \item \texttt{fix\_ui\_placeholders.py} 作为最后保险,确保仓库无遗留占位符。
\end{itemize}

\section{方法论抽象}

以下模式可迁移到任何结构化批量翻译或替换任务:
\begin{enumerate}
  \item \textbf{可逆标记(Tagging)}:为文本加唯一占位符,建立映射。
  \item \textbf{大文件拼接(Concatenate)}:将所有带标记文件拼成 LLM 输入,解决上下文与截断问题。
  \item \textbf{LLM 翻译(Translate)}:分块调用模型,统一输出格式(如 sed/JSON)。
  \item \textbf{结构化应用(Apply)}:解析替换结果,遍历原文件结构精准写回。
  \item \textbf{标记清理(Cleanup)}:移除占位符,必要时全局扫描兜底。
  \item \textbf{校验与监控(Verify \& Monitor)}:格式校验器、token 监控,保障质量与成本可见性。
\end{enumerate}

\section{可迁移示例}

\begin{longtable}{p{3cm} p{4cm} p{7cm}}
\hline
场景 & 文件格式 & 标记节点示例 \\
\hline
Vue i18n & \texttt{.vue} / \texttt{.ts} & \verb|t('¥0000001¥ Hello')| \\
JSON 配置 & \texttt{.json} & \verb|"title": "¥0000023¥ My Title"| \\
Markdown 文档 & \texttt{.md} & \verb|> ¥0000042¥ Some sentence| \\
Slicer XML 描述 & \texttt{.xml} & 同本案例 \\
\hline
\end{longtable}

\section{后续改进方向}

\begin{enumerate}
  \item \textbf{增量翻译}:借助 \texttt{git diff} 仅重标记新增英文文本。
  \item \textbf{Prompt 动态优化}:根据历史翻译质量自动调整 style guide。
  \item \textbf{多语言支持}:占位符阶段记录目标语言代码,支持一次生成多语种。
  \item \textbf{CI 集成}:在 PR workflow 中自动执行 \lstinline|make translate| 并回传结果。
\end{enumerate}

\section{更多应用设想}

除本报告所述 UI 翻译外,该批量标记–清洗–回写的方法论可延伸至更广阔的场景,列举如下,供脑洞参考:
\begin{enumerate}
  \item \textbf{自动脱敏 / 隐私保护}\footnote{如 GDPR 合规}:对企业文档批量插入占位符以标记姓名、地址、身份证号等敏感实体,随后调用 LLM 进行假名化或消除,再写回原格式(Word/Markdown/PDF 元数据)。
  \item \textbf{医疗影像 DICOM 头一致化}:批量扫描 DICOM 文件,使用占位符标记医院、日期等字段,经脚本统一重写以满足多中心数据集匿名化与字段映射。
  \item \textbf{金融报表数字校验}:对 Excel/CSV 批量添加占位符,LLM 校对币种、千分位与四舍五入规则,再写回清洗数据。
  \item \textbf{软件遗留代码现代化}:在上百万行 C++/Python 代码中占位符标记过时 API,借助 LLM 给出新 API 替换语句,自动补丁回写并生成迁移报告。
  \item \textbf{合规术语统一}:大型企业内部多语种政策文件,通过占位符标注术语,利用 LLM 统一术语翻译并维护术语库。
  \item \textbf{游戏文本本地化 + 文化化}\footnote{Culturalization}:RPG 游戏剧情脚本加占位符,LLM 根据地区文化自动改写梗、换装置之后写回脚本文件。
  \item \textbf{多模态字幕同步}:电影字幕 \texttt{.srt} 文件中先插标记,再用 LLM 将翻译与时轴长度匹配(字符数压缩),回写保证口型同步。
  \item \textbf{日志级别重构}:在微服务日志中占位符标记 INFO/DEBUG 字符串,LLM 根据运行频率与运维策略调整为 WARN/ERROR 等级并回写配置。
  \item \textbf{知识库 SEO 优化}:批量占位符插入网页标题与 meta 描述,LLM 生成关键词优化后的文本并回写 HTML/Markdown。
  \item \textbf{VR/AR 场景提示词转换}:对互动脚本批量标记交互提示,LLM 根据设备(VR/AR/移动)自动改写提示词,保持沉浸感。
  \item \textbf{跨平台快捷键映射}:批量占位符标记 Windows/Mac 键位提示,LLM 根据目标平台自动转换为 Cmd/Control 等符号并回写帮助文档。
  \item \textbf{论文参考文献格式统一}:在 BibTeX/EndNote 导出中插入占位符,LLM 统一期刊缩写与 DOI 链接,回写生成期刊规范格式。
  \item \textbf{API 文档示例代码多语言生成}:占位符标记 Python 示例,LLM 生成 JavaScript/Go 等语言对等示例并插入到 Markdown 文档中。
  \item \textbf{智能重写 commit message}:批量标记简短 commit,LLM 依据 Conventional Commits 规范生成更具信息量的描述并回写 git 历史(rebase)。
  \item \textbf{IoT 设备配置国际化}:在大规模设备固件 JSON 中插入占位符,LLM 将用户提示翻译为当地语言并考虑单位制转换。
  \item \textbf{合同条款风险扫描与替换}:标记潜在高风险条款,LLM 给出更友好措辞或提示并将修订版插回 Word/Markdown 合同。
  \item \textbf{曲谱移调与和弦替换}:在 MusicXML 文件标记和弦序列,LLM 根据歌手音域自动移调与简化和弦后写回曲谱。
  \item \textbf{化学分子命名标准化}:在科研数据 CSV 中占位符标记分子俗名,LLM 替换为 IUPAC 系统命名。
  \item \textbf{OpenAPI 版本升级}:批量标记旧版 Swagger YAML 中弃用字段,LLM 生成 v3 等效字段并迁移示例。
  \item \textbf{教育测验题干难度调节}:在题库 JSON 标记题干,LLM 根据 Bloom 认知层级重写题目以生成多难度版本。
  \item \textbf{社交媒体敏感词过滤}:占位符标记潜在敏感词汇,LLM 根据地区法律自动替换或打码。
  \item \textbf{古籍 OCR 纠错}:在批量 OCR 结果中占位符标记低置信度字符,LLM 参考上下文进行智能纠错。
  \item \textbf{电商商品标题优化}:在商品 CSV 标题列插入占位符,LLM 生成长尾关键词并保持字符限制。
  \item \textbf{机器学习特征命名规范化}:在代码与配置中占位符标记特征名,LLM 统一蛇形/驼峰命名并同步到文档。
  \item \textbf{变更日志自动分类}:批量标记 release note 行,LLM 根据语义分类为 Feature/Bugfix/Perf 等,并输出结构化 CHANGELOG。
  \item \textbf{多语种字幕情绪标注}:在剧集字幕标记句子,LLM 推断情绪标签(喜/怒/哀)并作为嵌入写回。
  \item \textbf{3D 打印 G-code 参数优化}:占位符标记关键速度/温度指令,LLM 根据材料建议参数,批量替换生成新 G-code。
  \item \textbf{CLI 帮助信息一致化}:在各微服务 CLI 输出占位符标记帮助文本,LLM 统一格式与示例。
  \item \textbf{多地区法律条款对照}:在法规数据库占位符标记条款编号,LLM 生成跨国对照摘要并回写 JSON。
  \item \textbf{网站可达性增强}:批量标记 HTML img 缺失 alt 属性,LLM 根据上下文生成描述并自动补全。
\end{enumerate}

\section{源码字符串翻译流水线}

本节说明如何借助前一章的“占位符 \textrightarrow LLM \textrightarrow 回写”范式,把 \textbf{Python / C++ 源码中的可翻译字符串}(模块 title、Qt \lstinline|tr()| 等)批量翻译成中文。

\subsection{新增脚本说明}
\begin{longtable}{p{4cm} p{11cm}}
\hline
脚本 & 作用 \\
\hline
\texttt{generate\_code\_placeholders.py} & 递归扫描指定后缀(\texttt{.py .c .cpp .h ...}),匹配三类正则:\verb|_('text')|、\verb|self.parent.title = 'text'|、\verb|::tr("text")|,插入占位符并生成 \texttt{tmp\_code/} 与 \texttt{all\_strings\_code.txt}。\\
\texttt{apply\_code\_translations.py} & 解析 LLM 输出的 \texttt{replace.sh},将中文写回 \texttt{tmp\_code/},随后同步到源码目录。\\
\texttt{fix\_placeholders.py} & 通用占位符清除器;支持 \texttt{--ext},自适应 XML 或纯文本。\\
\hline
\end{longtable}

\subsection{Makefile 一键命令}
\begin{lstlisting}
$ make -C script code_translate   # 插入占位符 -> LLM 翻译 -> 写回源码
$ make -C script code_fix         # 清理 UI 与源码中残留的 ¥XXXXXXXXXX¥ 占位符
\end{lstlisting}

脚本接受与 UI 流水线相同的环境变量 \texttt{MODEL}/\texttt{CHUNK\_SIZE}/\texttt{BASE\_URL},可直接在 CI/CD 环境注入。

\subsection{参数化与可扩展性}
\begin{itemize}
  \item \textbf{可扩展正则}:通过 \lstinline|--pattern "(?P<prefix>....)(?P<text>...)(?P<suffix>...)"| 追加匹配模式;命名捕获组 \texttt{text} 为必需。
  \item \textbf{多后缀支持}:\lstinline|--ext py,rs,go| 可轻松覆盖 Rust、Go 等新语言。
  \item \textbf{代理与分块}:沿用 UI 流水线的 \texttt{--base\_url} 与 \texttt{--chunk\_size},大文件可按需拆分。
\end{itemize}

此后若 UI 与源码均需翻译,可先执行 \lstinline|make translate| 完成 UI 部分,再运行 \lstinline|make code_translate| 同步源码文本;最终用 \lstinline|make code_fix| 统一清除占位符。

\end{document} 